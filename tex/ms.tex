% !TeX root = ./ms.tex
\documentclass[modern]{aastex62}

% Load the corTeX style definitions
% !TeX root = ./ms.tex
% All the packages
\usepackage{url}
\usepackage{amsmath}
\usepackage{mathtools}
\usepackage{amssymb}
\usepackage{natbib}
\usepackage{graphicx}
\usepackage{calc}
\usepackage{etoolbox}
\usepackage{xspace}
\usepackage[T1]{fontenc} % https://tex.stackexchange.com/a/166791
\usepackage{textcomp}
\usepackage{ifxetex}
\ifxetex
  \usepackage{fontspec}
  \defaultfontfeatures{Extension = .otf}
\fi
\usepackage{fontawesome}
\usepackage{listings}
\usepackage{nicefrac}
%\usepackage{bm}
\usepackage{booktabs}
\usepackage{longtable}

% Shorthand for this paper
\newcommand{\starry}{\textsf{starry}\xspace}
\newcommand{\cpp}{\textsf{C++}\xspace}
\newcommand{\theano}{\textsf{theano}\xspace}
\newcommand{\pymcthree}{\textsf{pymc3}\xspace}
\newcommand{\starryprocess}{\textsf{starry\_process}\xspace}
\newcommand{\Python}{\textsf{Python}\xspace}
\newcommand{\xxx}[1]{{\color{red}#1}}
\newcommand{\quadquad}{\quad\quad\quad\quad}

% References to text content
\newcommand{\documentname}{\textsl{article}}
\newcommand{\figureref}[1]{\ref{fig:#1}}
\newcommand{\Figure}[1]{Figure~\figureref{#1}}
\newcommand{\figurelabel}[1]{\label{fig:#1}}
\renewcommand{\eqref}[1]{\ref{eq:#1}}
\newcommand{\Eq}[1]{Equation~(\eqref{#1})}
\newcommand{\eq}[1]{\Eq{#1}}
\newcommand{\eqalt}[1]{Equation~\eqref{#1}}

% Add code, proof, and animation hyperlinks
\definecolor{linkcolor}{rgb}{0.1216,0.4667,0.7059}
\definecolor{testpasscolor}{rgb}{0.13333333,0.5254902,0.22745098}
\definecolor{testmissingcolor}{rgb}{1.0,0.88,0.30}
\definecolor{testfailcolor}{rgb}{0.79607843,0.14117647,0.19215686}
\newcommand{\codeicon}{{\color{linkcolor}\faCloudDownload}}
\newcommand{\testmissingicon}{{\color{testmissingcolor}\faQuestion}}
\newcommand{\testpassicon}{{\color{testpasscolor}\faCheck}}
\newcommand{\testfailicon}{{\color{testfailcolor}\faTimes}}
\newcommand{\codelink}[1]{\href{https://github.com/rodluger/starry_process/blob/8cc07b6268e28ce42b1ce63cb09147746d49c236/tex/figures/#1.py}{\codeicon}\,\,}
\newcommand{\animlink}[1]{\href{https://github.com/rodluger/starry_process/blob/8cc07b6268e28ce42b1ce63cb09147746d49c236/tex/figures/#1.gif}{\animicon}\,\,}
\newcommand{\prooflink}[1]{\href{https://github.com/rodluger/starry_process/blob/8cc07b6268e28ce42b1ce63cb09147746d49c236/tex/tests/#1.py}{\raisebox{-0.1em}{\input{tests/#1.tex}}}}
\newcommand{\cilink}[1]{\href{https://dev.azure.com/rodluger/starry_process/_build}{#1}}


% Define a proof environment for open source equation proofs
\newtagform{eqtag}[]{(}{)}
\newcommand{\currentlabel}{None}
\newenvironment{proof}[1]{%
  \ifstrempty{#1}{%
    \renewtagform{eqtag}[]{\raisebox{-0.1em}{{\testmissingicon}}\,(}{)}%
  }{%
    \renewtagform{eqtag}[]{\prooflink{#1}\,(}{)}%
  }%
  \usetagform{eqtag}%
  \renewcommand{\currentlabel}{#1}
  \align%
}{%
  \endalign%
  \renewtagform{eqtag}[]{(}{)}%
  \usetagform{eqtag}%
  \message{<<<\currentlabel: \theequation>>>}%
}

% Define the `oscaption` command for open source figure captions
\newcommand{\oscaption}[2]{\caption{#2 \codelink{#1}}}

% Code examples
\definecolor{codegreen}{rgb}{0,0.6,0}
\definecolor{codegray}{rgb}{0.5,0.5,0.5}
\definecolor{codepurple}{rgb}{0.58,0,0.82}
\definecolor{backcolour}{rgb}{0.95,0.95,0.95}
\lstdefinestyle{mystyle}{
  backgroundcolor=\color{backcolour},
  commentstyle=\color{codegreen},
  keywordstyle=\color{magenta},
  numberstyle=\tiny\color{codegray},
  stringstyle=\color{codepurple},
  basicstyle=\small\ttfamily,
  breakatwhitespace=false,
  breaklines=true,
  captionpos=b,
  keepspaces=true,
  numbers=left,
  numbersep=5pt,
  showspaces=false,
  showstringspaces=false,
  showtabs=false,
  tabsize=2,
  aboveskip=1em,
  belowskip=1em,
  keywords=[2]{map},
  keywordstyle=[2]{\color{black!80!black}},
  upquote=true
}
\lstset{style=mystyle}

% Typography obsessions
\setlength{\parindent}{3.0ex}
\renewcommand\quad{\hskip\fontdimen3\font}

% https://tex.stackexchange.com/a/184474
\usepackage{stackengine,scalerel}
\def\lnlam{\ThisStyle{\ensurestackMath{\stackon[-2.4\LMpt]{%
        \SavedStyle\lambda}{\kern-.5pt\kern\LMpt\rule{1\LMex}{.25pt+.15\LMpt}}}}}


% Load custom style
% Packages
\usepackage{fontspec}
\usepackage{unicode-math}
\usepackage{xifthen}
\usepackage{stackengine}
\usepackage{tabstackengine}
\usepackage{array}
\usepackage{upgreek}
\usepackage[bbgreekl]{mathbbol}
\usepackage{afterpage}

% Misc. macros
\newcommand{\LMAX}{15\xspace}

% Integrals
\newcommand{\dd}{\ensuremath{\text{d}}}

% Special functions
\newcommand{\sgn}{{\text{sgn}}}
\newcommand{\atantwo}{{\text{arctan2}}}
\newcommand{\imag}{{\ensuremath{\mathbb{i}}}}

% Cartesian unit vectors
\newcommand{\xhat}{\ensuremath{\pmb{\hat{x}}}\xspace}
\newcommand{\yhat}{\ensuremath{\pmb{\hat{y}}}\xspace}
\newcommand{\zhat}{\ensuremath{\pmb{\hat{z}}}\xspace}

% Other
\DeclarePairedDelimiter\ceil{\lceil}{\rceil}
\DeclarePairedDelimiter\floor{\lfloor}{\rfloor}

% Inverse diagonal dots
\makeatletter
\def\Ddots{\mathinner{\mkern1mu\raise\p@
\vbox{\kern7\p@\hbox{.}}\mkern2mu
\raise4\p@\hbox{.}\mkern2mu\raise7\p@\hbox{.}\mkern1mu}}
\makeatother

% Bibliography
\bibliographystyle{aasjournal}

\usepackage{etoolbox}
\makeatletter % we need to patch \env@cases that has @ in its name
\patchcmd{\env@cases}{\quad}{\qquad\qquad}{}{}
\makeatother

\usepackage{enumitem}

% Begin!
\begin{document}

% Title
\title{%
    \textbf{
        Interpretable Gaussian Processes for Stellar Light Curves
    }
}

% Author list
\author[0000-0002-0296-3826]{Rodrigo Luger}\altaffiliation{Flatiron Fellow}
\email{rluger@flatironinstitute.org}
\affil{Center~for~Computational~Astrophysics, Flatiron~Institute, New~York, NY}
\affil{Virtual~Planetary~Laboratory, University~of~Washington, Seattle, WA}
%

\keywords{methods: analytic}

% \begin{abstract}
%     Abstract here.
%     %
%     \href{https://github.com/rodluger/starry_process}{\color{linkcolor}\faGithub}
% \end{abstract}

%
\section{The Spot Expansion}
\label{sec:spot}
%
We adopt the following expression for the spherical harmonic coefficient
of degree $l$ and order $m$ in the expansion of a spot at
$\theta = \varphi = 0$:
%
\begin{align}
    \label{eq:ylm0}
    y_{lm}(\delta, r, \theta = 0, \varphi = 0) =
    \begin{cases}
        1 - \dfrac{\delta c r}{2 (1 + c r)}
         & l = m = 0    \\[2em]
        -\dfrac{\delta c r \left( 2 + c r \right)}
        {2 \sqrt{2l + 1} (1 + c r)^{l + 1}}
         & l > 0, m = 0 \\[2em]
        0
         & m \ne 0
    \end{cases}
\end{align}
%
where $\delta \in [-\infty, 1]$ is the fractional decrease in the brightness
at the center of the spot and $r \in [0, 1]$ is a normalized spot radius.
The quantity $c$ is a normalization constant for the radius (see below).

The expression in Equation~(\ref{eq:ylm0}) is convenient because it satisfies
three important properties:
%
\begin{enumerate}[itemsep=2pt,parsep=1pt,label=\textbf{\arabic*}]
    \item The surface intensity monotonically increases away from the spot center
    \item The surface intensity at the spot center is $1 - \delta$
    \item The surface intensity at the antipode of the spot center is unity
\end{enumerate}
%
These properties may be demonstrated by considering the expression for the
surface intensity at a given point $(\theta, \varphi)$:
%
\begin{align}
    I(\theta, \varphi)
     & =
    \sum\limits_{l=0}^\infty \sum\limits_{m=-l}^l
    y_{lm} Y_{lm}(\theta, \varphi) \\
     & = \sum\limits_{l=0}^\infty
    y_{l0} \sqrt{2l + 1} P_l(\cos\theta)
\end{align}
%
where $P_l$ is the Legendre polynomial of degree $l$ and we have implicitly
assumed a normalization such that the integral of our expansion over
the unit sphere is $4\pi$.
Combining this with Equation~(\ref{eq:ylm0}) and rearranging, we may write
%
\begin{align}
    \label{eq:Igen}
    I(\theta, \varphi) =
    1 + \dfrac{\delta c r}{2}
    -
    \dfrac{\delta c r \left( 2 + c r \right)}{2 (1 + c r)}
    \sum\limits_{l=0}^\infty \left(\dfrac{1}{1 + c r}\right)^l P_l(\cos\theta)
    \quad.
\end{align}
%
The summation in Equation~(\ref{eq:Igen}) has a closed-form expression in
terms of the generating function of the Legendre polynomials:
%
\begin{align}
    \label{eq:gen}
    \sum\limits_{l=0}^\infty t^l P_l(\cos\theta) = \frac{1}{\sqrt{1 + t^2 - 2 t \cos\theta}}
    \quad,
\end{align}
%
so we may express the intensity in the fairly simple form
%
\begin{align}
    \label{eq:Ifinal}
    I(\theta, \varphi) = A - \frac{B}{\sqrt{C - \cos\theta}}
    \quad,
\end{align}
%
where
%
\begin{align}
    A & = 1 + \dfrac{\delta c r}{2}                      \\
    B & = \delta c r (2 + cr) \sqrt{\dfrac{1}{8 + 8 cr}} \\
    C & = \dfrac{1 + (1 + c r)^2}{2 + 2 c r}
\end{align}
%
are positive constants.

Differentiating Equation~(\ref{eq:Ifinal}) with respect to $\theta$, we have
%
\begin{align}
    \label{eq:Ideriv}
    \dfrac{\mathrm{d}I(\theta, \varphi)}{\mathrm{d}\theta} & =
    -\frac{B\sin\theta}{2(C - \cos\theta)^\frac{3}{2}}
    \quad,
\end{align}
%
which is zero only for $\theta = 0$ (for which $I(\theta, \varphi)$ is
minimized) and $\theta = \pi$ (for which it is maximized). The intensity
therefore increases monotonically from the spot center to the antipode,
as stated in \textbf{1}. The value at the minimum is
%
\begin{align}
    I_{\mathrm{min}} & = A - \dfrac{B}{\sqrt{C - 1}} \\
                     & = 1 - \delta
    \quad,
\end{align}
%
as stated in \textbf{2}, and the value at the maximum is
%
\begin{align}
    I_{\mathrm{max}} & = A - \dfrac{B}{\sqrt{C + 1}} \\
                     & = 1
    \quad,
\end{align}
%
as stated in \textbf{3}.

% At the spot center, $\theta = 0$, so $\cos\theta = 1$. Since $P_l(1) = 1$ for
% all $l$, we have
% %
% \begin{align}
%     I(0)
%      & =
%     \sum\limits_{l=0}^\infty y_{l0} \sqrt{2l + 1} \\
%      & =
%     1 - \dfrac{\delta c r}{2 (1 + c r)}
%     -
%     \dfrac{\delta c r (2 + c r)}{2 (1 + c r)}
%     \sum\limits_{l=1}^\infty
%     \dfrac{1}
%     {(1 + c r)^{l}}                               \\
%      & =
%     1 -
%     \delta
%     \left[
%         \dfrac{c r}{2 (1 + c r)}
%         - \dfrac{2 + c r}{2(1 + c r)}
%         \right]                                   \\
%      & = 1 - \delta
%     \quad,
% \end{align}
% %
% as stated in \textbf{2}, where we made use of the fact that
% %
% \begin{align}
%     \sum\limits_{l=1}^\infty
%     \dfrac{1}
%     {(1 + x)^{l}} = \frac{1}{x}
%     \quad.
% \end{align}
% %
% Finally, at the antipode, $\theta = \pi$,
% $\cos\theta = -1$, and $P_l(\cos\theta) = (-1)^l$, so
% %
% \begin{align}
%     I(\pi)
%      & =
%     \sum\limits_{l=0}^\infty y_{l0} (-1)^l \sqrt{2l + 1} \\
%      & =
%     1 - \dfrac{\delta c r}{2 (1 + c r)}
%     -
%     \dfrac{\delta c r (2 + c r)}{2 (1 + c r)}
%     \sum\limits_{l=1}^\infty
%     \dfrac{(-1)^l}
%     {(1 + c r)^{l}}                                      \\
%      & =
%     1 - \dfrac{\delta c r}{2 (1 + c r)}
%     +
%     \dfrac{\delta c r}{2 (1 + c r)}                      \\
%      & = 1
%     \quad,
% \end{align}
% as stated in \textbf{3}, where we made use of the fact that
% %
% \begin{align}
%     \sum\limits_{l=1}^\infty
%     \dfrac{(-1)^l}
%     {(1 + x)^{l}} = -\frac{1}{2 + x}
%     \quad.
% \end{align}
%
%\bibliography

\end{document}
